\chapter{Methodology}

\section{Participants}

For this study, I recruited six Icelandic speakers who were living in York (UK) when the recordings were made.
%TODO Ethic statement
Recruitment was done through University channels, the Icelandic Embassy in London and the York Anglo Scandinavian Society.
All the participants were native speakers of Icelandic, above 18 years old and claimed to have normal hearing and speech abilities.
The information on each participant is given in \Cref{t:participants}.
%TODO Explain table
Participant JR had to be excluded from the analysis since he misunderstood the task, while part of participant SHG's task was lost due to a technical fault in the recording equipment.


\ctable[caption = Information on participants,
label = t:participants,
]{ccccclc}{}{
\FL
id & sex    & age & born & city & languages & abroad \ML
TT & female & 24 & Reykjavik & Reykjavik & English, Danish, German  & Yes \NN
BRS & female & 25 & Hofn      & Hofn      & Danish, English, Spanish & Yes \NN
BTE & female & 27 & Reykjavik & Reykjavik & English, Danish          & Yes \NN
JJ & female & 46 & Reykjavik & Kopavogur & English, Danish          & Yes \NN
SHG & male   & 25 & Selfoss   & Selfoss   & English                  & No  \NN
JR & male   & 66 & Reykjavik & York      & English                  & Yes \LL
}

\section{Materials}

The material used in the task consisted of a list of Icelandic words (the ``target words'') with the following forms: (C)VCC (monosyllabic) and (C)VCCV (bisyllabic).
The list of target words is given in \Cref{c:appendix}.
The target words were selected so as to control for as many of the following aspects as possible: phonation, manner and place of articulation of consonants following the target vowel; height and frontness of the target vowel; phonation, manner and place of articulation of consonants preceding the target vowel; and height and frontness of the eventual word-final vowel.
Control over these parameters was prioritised according to the order in which they were presented here.
Unfortunately, obtaining a well controlled word list proved to be extremely difficult and several compromises have been made.

%TODO explain words form and how many!


\section{Procedure}

The target words were embedded in the frame sentence \textit{Segðu \_\_ aftur}, `Say \_\_ again.'
This sentence was chosen with the aid of one of the participants so as to control for naturalness, number of syllables and phonetic contexts preceding and following the target word, and phrase stress.
The participants were asked to read aloud the sentences with the target words shown on a computer screen.
They were advised to speak as naturally as possible, while keeping the same volume and pace.
They did not familiarised themselves with the word list before starting the task.
The decision of not showing the words beforehand was made to reduce the speakers' control over their speech.
The task was presented through the software PyschoPy \citep{peirce2009}, on a Apple MacBook Pro (mid 2014 model).
Each sentences was shown three times consecutively and the order of appearance was randomised across subjects.
The reading task was self-paced; the participant read a sentence shown on the screen and moved to the next sentence when ready by pressing the space bar.

The informants were recorded using a high-fidelity headset microphone plugged into a Zoom H4n Handy Recorder.
The audio files were encoded using the \texttt{.wav} format at a sampling rate of 44 kHz (16-bit).







