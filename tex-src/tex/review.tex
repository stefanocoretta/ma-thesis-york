\chapter{Literature review}
\label{c:review}

\section{Laryngeal}



\section{The effect of aspiration on vowel duration}

Vowel duration has been reported in the literature to correlate with the presence vs. absence of aspiration in the following consonant.
In particular, \citet{maddieson1976} and \citet{durvasula2012} found that vowels followed by aspirated consonants in Hindi are longer than vowels followed by non-aspirated consonants.
In the following paragraphs, I will briefly introduce the system of laryngeal oppositions of Hindi.
I will then review some of the findings concerning the aspiration effect and the major theories regarding the cause of this phenomenon.

%TODO citation for Hindi!
The consonantal system of Hindi is based on a four-way opposition of laryngeal contrasts.
For each place of articulation, there are a voiceless unaspirated, a voiced unaspirated, a voiceless aspirated and (breathy) voiced aspirated stop: for example, [t], [d], [tʰ], [dʱ].
The voiceless aspirated stops (like [tʰ]) are similar to the aspirated stops of English: a relatively long VOT follows the release of the occlusion.
The voiced counterpart (like [dʱ]) is normally voiced throughout the closure and the aspiration is characterised by breathy voicing.
\citet{maddieson1976} found that vowels followed by voiced and voiceless aspirated stops (like in [kaːd] `embroider' and [kaːtʰ] `wood') were of equal length but longer than vowels followed by voiceless stops (like in [kaːt] `cut').
Moreover, vowels followed by voiced aspirated stops (like in [saːdʱ] `balance') were even longer than voiced and voiceless aspirated stops.
\Cref{t:hindi} shows the mean duration of vowels before the four alveolar stops as reported by \citet[47]{maddieson1976}.

%TODO say in note that phonemic is different
\ctable[caption = Mean duration of vowels in Hindi before stops.,
label = t:hindi
]{ll}{}{
\FL
consonant & vowel duration (msec) \ML
/t/       & 160 \NN 
/d/       & 184.5 \NN
/tʰ/      & 184.75 \NN
/dʰ/      & 196 \LL
}









