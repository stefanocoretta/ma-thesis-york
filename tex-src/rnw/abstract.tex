\abstractintoc

\begin{abstract}
This dissertation dealt with the relation between vowel duration and aspiration in consonants.
Previous studies showed that vowels in Hindi are longer when followed by aspirated consonants (like in the word \textit{sāth} ``companionship''), while are shorter if a non-aspirated consonant follows (like in  \textit{sāt} ``seven'').
Such phenomenon has been called the ``aspiration effect.''
No explanation has been given to this effect and the research conducted to clarify it only focussed on post-aspiration.
The finding that post-aspiration in a consonant lengthens the preceding vowel implies two possible prediction for what concerns pre-aspiration: pre-aspiration could behave like post-aspiration, causing vowels to lengthen; or it could instead make them shorter.
If the aspiration effect is caused by the relative phasing of glottal spreading, an early timing of glottal abduction (as in pre-aspiration) should have a shortening effect.
I called this the ``timing hypothesis.''
I carried out a data collection with 5 natives speakers of Icelandic, a language that has contrastive aspiration in stops and sonorants, to test the timing hypothesis.
I then extracted the duration of vowels followed by aspirated versus non-aspirated consonants.
The results showed that vowels before aspirated consonants (like in Icelandic \textit{takka} `key' [tʰaʰka]) were significantly shorter than vowels followed by non-aspirated consonants (like in \textit{kagga} `barrel' [kʰakka]).
On average, vowels followed by aspirated consonants were 30 milliseconds longer than when followed by non-aspirated stops.
While such difference can be attributed to the different duration of aspirated and non-aspirated sonorants, geminate stops showed precisely the predicted difference in timing of glottal spread.
The beginning of the abduction gesture---necessary to sustain voicelessness in non-aspirated geminates and frication in pre-aspirated geminates---was timed significantly earlier in pre-aspirated stops.
The results thus indicate that, at least in Icelandic, the differences in vowel duration could be caused by differences in timing of laryngeal spread, thus provisionally validating the timing hypothesis.
\end{abstract}

\newpage
\begin{adjustwidth}{2cm}{}
\vspace*{4\baselineskip}
As more absolute universals ``bite the dust,'' we should not despair.
The challenges posed by universal tendencies arc just as great, if not greater, than those posed by absolute universals.
We must understand not only why most languages have a particular property, but why there are the rare exceptions that do not.
By meeting these challenges, we will ultimately have a deeper understanding of sound patterns as they reflect human potential in articulation, perception, and general cognition.

-- \citet[p. 272]{blevins2009}

\vfill
\end{adjustwidth}
\newpage








